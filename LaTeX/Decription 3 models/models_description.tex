\documentclass{article}

\usepackage{amsmath,amssymb}
\allowdisplaybreaks


\begin{document}

\section{Problem description}

Let $N=\{1,...,n\}$ be a set of players. Let $G=(V,E)$ be a complete and direct graph with $V$ and $E$ the sets
of nodes and edges respectively. Each edge $e \in E$ will be noted by a tuple, $e=(v,w,i)$, indicating that it 
connects the node $v\in V$ with the node $w \in V$ and that his owner is the agent $i\in N$. Note that exist as
many edge between any pair of nodes as agents. Each edge $e \in E$ will have a capacity $q_e$ and a cost for using it $c_e$.
The cost of an edge would be payed by his owner if the edge is used, regardless on which fraction of its capacity
is actually used. Let $\Theta$ be the set of commodities. Each commodity is noted by a tuple, $(o,t,i)$, where $o\in V$ is its origin
node, $t\in V$ is its destiny node, and $i$ is the agent who owns it. Each commodity $(o,t,i)$ has an integer size, $d_{(o,t,i)}$
and an associated revenue $r_{(o,t,i)}$ that its owner earns if the commodity is succesfully delivered from $o$ to $t$.

The commodities are unsplitable, i.e., can not be divided into different edges.

We will note by $E^i \subset E$ and $\Theta^i$ the sets of edges and commodities owned by player $i$.


\subsection{Single agent model}

The objective of each player $i \in N$ is to maximizes his/her payoff by delivering his/her commodities using 
his/her own edges.

Therefore, the problem which player $i\in N$ has to solve can be modeleg as an ILP:





\begin{subequations}
    \begin{alignat}{3}
        &  P_i: \max  & \hspace{22pt} \sum_{(o,t,i)\in \Theta^i} \sum_{e \in IE^i(t)}  f_e^{(o,t,i)} \cdot d_{(o,t,i)} \cdot r - \sum_{e'\in E^i} u_{e'}\cdot c_{e'} \hspace{40pt} &&   \label{eq:SingleAgentA}
    \end{alignat}
    \begin{alignat}{3}
        & \text{subject to}       & \sum_{e \in IE^i(z)} f_e^{(o,t,i)}-\sum_{e' \in OE^i(z)} f_{e'}^{(o,t,i)} & = 0,                                   && \forall\ z\in V\setminus\{o,t\},\nonumber\\[-1em]
        &                         &                                                                           &                                        && \forall\ (o,t,i)\in\Theta^i,  \label{eq:SingleAgentB}\\[1em]
        &                         & \sum_{e \in OE^i(o)} f_e^{(o,t,i)}                                        & \leq 1,                                && \forall\ (o,t,i)\in \Theta^i, \label{eq:SingleAgentC} \\
        &                         & \sum_{e \in OE^i(t)} f_e^{(o,t,i)}                                        & = 0,                                   && \forall\ (o,t,i)\in \Theta^i, \label{eq:SingleAgentD} \\
        &                         & \sum_{(o,t,i) \in \Theta^i} f_e^{(o,t,i)}\cdot d_{(o,t,i)}                & \leq u_e\cdot q_{(o,t,i)},\hspace{8pt} && \forall\ e \in E^i, \label{eq:SingleAgentE}  \\
        &                         & \sum_{v \in S} \sum_{w \in S} f_{(v,w)}^{(o,t,i)}                         & \leq |S| -1,                           && \forall\ S \subset V, \nonumber\\[-1em]
        &                         &                                                                           &                                        && \forall\ (o,t,i) \in \Theta^i, \label{eq:SingleAgentF}\\[1em]
        &                         & f_e^{(o,t,i)}                                                             & \in \{0,1\},                           && \forall\ e \in E^i,\nonumber\\
        &                         &                                                                           &                                        && \forall\ (o,t,i) \in \Theta^i, \label{eq:SingleAgentG} \\[1em]  
        &                         &  u_e                                                                      & \in \{0,1\},                           && \forall\ e \in E^i 
    \end{alignat}

\end{subequations}

Equation (\ref{eq:SingleAgentA}) maximizes the profits of agent $i$. 

Constraint (\ref{eq:SingleAgentB}) ensures that, for any commodity, the flow that enters and leaves transit nodes (neither the origin neither the terminal) is equal.


Constraints (\ref{eq:SingleAgentC}), together with (\ref{eq:SingleAgentG}), ensures that any commodity, in case of be send, is completely send and only through and edge from its origin node.

Constraint (\ref{eq:SingleAgentD}) ensures that none commodity is send from its terminal node. Constraint (\ref{eq:SingleAgentE}) ensures that the capacity of edges is not exceed.

Constraint (\ref{eq:SingleAgentF}) is a subtour elimination constraint.

We will note by $P_i^*$ the optimal solution of $P_i$.

\subsection{Different types of cooperation}

We will consider 3 different types of colaboration between the agents, which will differetiate by 
the number of decisions that are left to a central planner. 


\begin{table}[h!]
    \begin{tabular}{cc|ccc}
        & &      \multicolumn{3}{|c}{Level cooperation} \\\cline{3-5}
        & & Partial-1 & Partial-2 & Full \\ \hline
        AGENT & Activate edges & Yes & Yes & No \\
        '' & Route flow     & Yes & No & No \\\hline
        Central planner & Activate edges & No & No & Yes \\
        '' & Route flow & Yes & Yes & Yes
        \end{tabular}
    \end {table}

\subsubsection*{Partial-1 cooperation}


In the first type of cooperation, what we will call ``Partial-1 cooperation'', each agent $i\in N$ will solve its own ($P^i$) problem and 
then pass to the central planner the demands he was not able to serve and the not used capacity of the 
edges he has activated. Will the information from all the agents, the central planner will try to maximize
the profits of the grand coalition, routing the unserved commodities through the not used capacity
of the edges. If a commodity is route through an edge ow by a different agent, the owner of the commodity
will pay to the owner of the edge the proportional cost of the fraction of the capacity of th edge he is using.

Therefore, the central planner would be solving the following model:


\begin{subequations}
    \begin{alignat}{3}
        &  \Gamma_1: \max  & \sum_{(o,t,i) \in \Theta} \left[\sum_{e \in OE(o)}  f_e^{(o,t,i)} \cdot d_{(o,t,i)} \cdot r_{(o,t,i)} - \sum_{\substack{e \in E^j\colon \\ j\not = i}} f_e^{(o,t,i)} \cdot \frac{c_e}{q_e} \right] &&   \label{eq:Partial1CooperationA} 
    \end{alignat}
    \begin{alignat}{3}
        & \text{subject to}       & \sum_{e \in IE(z)} f_e^{(o,t,i)}-\sum_{e' \in OE(z)} f_{e'}^{(o,t,i)} & = 0,            && \forall\ z\in V\setminus\{o,t\},\nonumber\\[-1em]
        &                         &                                                                       &                 && \forall\ (o,t,i)\in\Theta,  \label{eq:Partial1CooperationB}\\[1em]
        &                         & \sum_{e \in OE(o)} f_e^{(o,t,i)}                                      & \leq 1,         && \forall\ (o,t,i)\in \Theta, \label{eq:Partial1CooperationC} \\
        &                         & \sum_{e \in OE(t)} f_e^{(o,t,i)}                                      & = 0,            && \forall\ (o,t,i)\in \Theta, \label{eq:Partial1CooperationD} \\
        &                         & \sum_{(o,t,i) \in \Theta} f_e^{(o,t,i)}\cdot d_{(o,t,i)}              & q_{(o,t,i)}     && \forall\ e \in E, \label{eq:Partial1CooperationE}  \\
        &                         & \sum_{v \in S} \sum_{w \in S} f_{(v,w)}^{(o,t,i)}                     & \leq |S| -1,    && \forall\ S \subset V, \nonumber\\[-1em]
        &                         &                                                                       &                 && \forall\ (o,t,i) \in \Theta, \label{eq:Partial1CooperationF}\\[1em]
        &                         & f_e^{(o,t,i)}                                                         & \in \{0,1\},    && \forall\ e \in E,\nonumber\\
        &                         &                                                                       &                 && \forall\ (o,t,i) \in \Theta, \label{eq:Partial1CooperationG} 
    \end{alignat}
\end{subequations}

where $E$ is the set of all the edges, $e$, activated by an agent and with some not used capacity, $q_e$; and 
$\Theta$ is the set of all the demands that the agent left unserved.

We will note by $\Gamma_1^*$ the optimal solution of  $\Gamma_1$, and by $\Gamma_1^*(i)$ the payoff which obtains
agent $i$. Thus, the final payoff
of each agent $i\in N$ when the Partial-1 cooperation is used, is $\Gamma_1^*(i)+P_i^*$


\subsubsection*{Partial-2 cooperation}

In the second type of cooperation that we study, ``Partial-2 cooperation'', all the agent will also start by solving
their $P^i$ problems. Once they have done it, they do not route any commodity, but they pass all the commodities to the central
planner, as well as all the edges they activate following their optimal solution for $(P^i)$. Then, as in ``Partial-1 cooperation'',
the central planner will try to maximize the profits of the Grand Coalition, but also ensuring that any agent will get a final payoff smaller
than what he will optain without cooperation. Therefore, to the model of ``Partial-1 cooperation'' is added the following constraint:


\begin{subequations}
    \begin{alignat}{3}
        &  \Gamma_2: \max  & \sum_{(o,t,i) \in \Theta} \left[\sum_{e \in OE(o)}  f_e^{(o,t,i)} \cdot d_{(o,t,i)} \cdot r_{(o,t,i)} - \sum_{\substack{e \in E^j\colon \\ j\not = i}} f_e^{(o,t,i)} \cdot \frac{c_e}{q_e} \right] &&   \label{eq:Partial2CooperationA} 
    \end{alignat}
    \begin{alignat}{3}
        & \text{subject to}       & \sum_{e \in IE(z)} f_e^{(o,t,i)}-\sum_{e' \in OE(z)} f_{e'}^{(o,t,i)} & = 0,            && \forall\ z\in V\setminus\{o,t\},\nonumber\\[-1em]
        &                         &                                                                       &                 && \forall\ (o,t,i)\in\Theta,  \label{eq:Partial2CooperationB}\\[1em]
        &                         & \sum_{e \in OE(o)} f_e^{(o,t,i)}                                      & \leq 1,         && \forall\ (o,t,i)\in \Theta, \label{eq:Partial2CooperationC} \\
        &                         & \sum_{e \in OE(t)} f_e^{(o,t,i)}                                      & = 0,            && \forall\ (o,t,i)\in \Theta, \label{eq:Partial2CooperationD} \\
        &                         & \sum_{(o,t,i) \in \Theta} f_e^{(o,t,i)}\cdot d_{(o,t,i)}              & q_{(o,t,i)}     && \forall\ e \in E, \label{eq:Partial2CooperationE}  \\
        &                         & \sum_{v \in S} \sum_{w \in S} f_{(v,w)}^{(o,t,i)}                     & \leq |S| -1,    && \forall\ S \subset V, \nonumber\\[-1em]
        &                         &                                                                       &                 && \forall\ (o,t,i) \in \Theta, \label{eq:Partial2CooperationF}\\[1em]
        &                         & \Gamma_2(i)                                                           & \geq P_i^*,     && \forall\ i\in N \label{eq:Partial2CooperationG}\\
        &                         & f_e^{(o,t,i)}                                                         & \in \{0,1\},    && \forall\ e \in E,\nonumber\\
        &                         &                                                                       &                 && \forall\ (o,t,i) \in \Theta, \label{eq:Partial2CooperationH}
    \end{alignat}
\end{subequations}

Constraint (\ref{eq:Partial2CooperationG}) ensures that the flow is routed in such a way that every agent ends up with a payoff equal or greater than what (s)he could
optain without cooperation.

\subsubsection*{Full cooperation}


The final type of cooperation we study, ``Full cooperation'', is the case where all the agent pass to the central planner all their commodities,
and let the central planner to decide which edges to activate and how to route the commodities.
As in ``Partial-2 cooperation'', the central planner must ensure that any agent end up with a payoff
smaller than what (s)he would have obtained without cooperation.

\begin{subequations}
    \begin{alignat}{3}
        &  P_N: \max  & \hspace{22pt} \sum_{(o,t,i)\in \Theta} \sum_{e \in IE(t)}  f_e^{(o,t,i)} \cdot d_{(o,t,i)} \cdot r - \sum_{e'\in E} u_{e'}\cdot c_{e'} \hspace{40pt} &&   \label{eq:FullCooperationA}
    \end{alignat}
    \begin{alignat}{3}
        & \text{subject to}       & \sum_{e \in IE(z)} f_e^{(o,t,i)}-\sum_{e' \in OE(z)} f_{e'}^{(o,t,i)} & = 0,                                   && \forall\ z\in V\setminus\{o,t\},\nonumber\\[-1em]
        &                         &                                                                           &                                        && \forall\ (o,t,i)\in\Theta,  \label{eq:FullCooperationB}\\[1em]
        &                         & \sum_{e \in OE(o)} f_e^{(o,t,i)}                                        & \leq 1,                                && \forall\ (o,t,i)\in \Theta, \label{eq:FullCooperationC} \\
        &                         & \sum_{e \in OE(t)} f_e^{(o,t,i)}                                        & = 0,                                   && \forall\ (o,t,i)\in \Theta, \label{eq:FullCooperationD} \\
        &                         & \sum_{(o,t,i) \in \Theta} f_e^{(o,t,i)}\cdot d_{(o,t,i)}                & \leq u_e\cdot q_{(o,t,i)},\hspace{8pt} && \forall\ e \in E, \label{eq:FullCooperationE}  \\
        &                         & \sum_{v \in S} \sum_{w \in S} f_{(v,w)}^{(o,t,i)}                         & \leq |S| -1,                           && \forall\ S \subset V, \nonumber\\[-1em]
        &                         &                                                                           &                                        && \forall\ (o,t,i) \in \Theta, \label{eq:FullCooperationF}\\[1em]
        &                         & P_N(i)                                                           & \geq P_i^*,     && \forall\ i\in N \label{eq:FullCooperationG}\\
        &                         & f_e^{(o,t,i)}                                                             & \in \{0,1\},                           && \forall\ e \in E,\nonumber\\
        &                         &                                                                           &                                        && \forall\ (o,t,i) \in \Theta, \label{eq:FullCooperationH} \\[1em]  
        &                         &  u_e                                                                      & \in \{0,1\},                           && \forall\ e \in E 
    \end{alignat}

\end{subequations}

\end{document}