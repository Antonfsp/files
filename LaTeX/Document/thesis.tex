\documentclass[review]{elsarticle}

\usepackage{amsmath,amssymb}
\usepackage{adjustbox}

\usepackage{lineno,hyperref}
\modulolinenumbers[5]
\allowdisplaybreaks


% This avoid a foot note about to which paper the article was submited
\makeatletter
\def\ps@pprintTitle{%
 \let\@oddhead\@empty
 \let\@evenhead\@empty
 \def\@oddfoot{}%
 \let\@evenfoot\@oddfoot}
\makeatother

%%%%%%%%%%%%%%%%%%%%%%%
%% `Elsevier LaTeX' style
\bibliographystyle{elsarticle-num}
%%%%%%%%%%%%%%%%%%%%%%%

\begin{document}

\begin{frontmatter}

\title{Intertemporal logistics collaboration \\[5pt]
                \normalsize{Msc Thesis}}

%% Group authors per affiliation:
\author{Antón de la Fuente suárez-Pumariega}
\address{a.delafuentesuarez-pumariega@student.maastrichtuniverisry.nl }

%% or include affiliations in footnotes:
\author{Maastricht Univeristy}



\begin{abstract}
Here goes the abstract
\end{abstract}

\begin{keyword}
Here\sep go \sep KEY \sep words
\end{keyword}

\end{frontmatter}

% Desactivation of numbering every 15 LINES
%\linenumbers

\section{Introduction}

During the last decades, logistics collaboration, and specially \emph{horizontal
logistics collaboration}, has gained the attention of many researchers, as it
has been proofed to be an effective strategy to improve the logistics chain, for
example, reducing both ecological \cite{BALLOT2010} or economical (find cite)
operational costs. By horizontal logistics collaboration we refer to the
cooperation between several companies or agents, who playing the same role in
the logistics chain, form coalitions or alliances to increase the efficiency of
their logistics operations. For example, in the case of the liner hipping
industry, by allowing other companies to use part of the capacity of the own
ships, increasing the asset utilization ratio \cite{AGARWAL2008175}.


The literature studying the horizontal logistics collaboration between agents
can be divided in two main streams depending on whether the collaboration it is
centralized or not. 

In the case of centralized systems, agents usually share in a pool all or
part of the demands that they have to serve as well as the assets they have
available. These assets can be vehicles in the case of transportation carriers,
but also inventory levels, employees, etc. Then a central planner, taking into
account all the demands and rources shared in the pool by the agents, solves the
problem maximizing the total payoff of the coalition. Finally, the benefits of
the cooperation are shared among the members of the coalition, usually using
some allocation rule which ideally satisfies some desiderable conditions in
terms of ``fairness'', as that the resulting allocation is in the \emph{core},
i.e., any subcoalition of the former one could obtain a higher payoff working
independently \cite{GONZALEZ2010}. A well known example of an allocation rule
fulfilling this conditions is the \emph{Shapley value} \cite{SHAPLEY1952}.

Some authors have argued in the centralized systems, the individual objectives
of the agents are not considered in favour of the coalition objective, while
actually the agents remain independent entities \cite{DEFRYN2018891}. Therefore,
these authors propose diferent mechanisms where both levels of objectives are
considered \cite{DEFRYN20191} \cite{VANOVERMEIRE2014125}.

In the other hand, in the descentralized systems there is not a central planner
who carries out the optimization and then allocates the benefits, but the agents
remain independent decision makers during the whole collaboration process.
Descentralized systems commonly operate based on auctions systems or
inverse-optimization techniques \cite{XIAOZHOU2013}. In the first, agents share
the part of their demands, an the other members of the coalition bid to be the
ones serving that demand on exchange of a recompense from the original owner of
the order. In the other hand, an example of a collaborative system based in a
inverse-optimization mechanism is the proposed in \cite{AGARWAL2008520}. On it,
the authors model a multicommodity flow problem, where agents own certain
fraction of the capacity of the edges of a network. Thus, the agents can
collaborate sharing the capacity of the edges with the members of the coalition,
who have to pay some price for using that edges. How much capacity the agents
own on each edge of the and which are their demand is assumed to be public
information. Using invere optimization, appropiate capacity exchange prices are
found, in such a way that if the agents adopt that prices policy, when each
agent selfishly optimizes his demands through the shared network, the resulting
flow will be the same that if a centralized approach was used. 

Some authors have argued that in the centralized systems, the individual
objectives of the agents are not considered in favour of the coalition
objective, while the agents actually remain independent entities
\cite{DEFRYN2018891}. Therefore, these authors propose diferent mechanisms where
both levels of objectives are considered through multi-objective approaches
\cite{DEFRYN20191} or by integrating the objectives of the agents as constraints
to the coalition problem \cite{VANOVERMEIRE2014125}.

In \cite{ANUPINDI2001} a novel framework to model descentralized systems is
proposed. Based in the concept of \emph{coopetition} \cite{BRANDENBURGER1996},
the authors proposed a 2-stage mechanism to model the collaboration of agents in
descentralized distribution systems. In the first stage, agents individually
takes strategic decisions, in their case, fixing their inventory levels. In the
seconds stage an allocation of the assets is obtained through cooperation, were
agents share their residual demands and inventory with the others, obtaining
extra profits. They propose sufficient conditions under which a Nash equilibrium
in pure strategiest exist for the first stage, and for which allocation rule in
the \emph{core} exists for the second stage.



In this paper we will compare the efficiency between a centralized system,
giving to central planner different degrees of decision power, and a
descentralized one, where we substite the central planner by a platform where
agents can share information, and through an iterative process of optimization
and information sharing, agents can achieve an equilibrium point from which any
of the agents would be intested to deviate from. The problem we
will use as example to implement this models is a multicommodity flow problem,
where each agent have to decide which edges to active on the network at certain
price, to then find an efficient flow trhough them which maximizes their
payoff.

In section blablbla. In section blalbla. Finally in sectioin blalblalba


\section{Problem statement: A decision making - muldicommodity flow problem}

As stated in the previous section, we will use a multicommodity flow problem
with a strategic planning stage to illustrate the different cooperation
mechanism we want to study in this paper. On this problem a set of agents have
to first decide which edges they want to activate on a network, with certain
costs associated. Then, the agents will have to route their commodities through
the active edges maximizing the revenues generated by the served commodities and
minimizing the costs associated with the activation of the edges. The agents
might collaborate among them sharing the capacity they do not use in the active
edges, allowing others to route their commodities through them at some price.

On the rest of this section we will introduce the specific characteristics of
these problem as well as the notation we will use. We will finish the section
modeling the problem that each agent would face in case he does not take part of
the collaboration.

Let $N=\{1,...,n\}$ be a set of agents. Let $G=(V,E)$ be a directed graph with
$V$ and $E$ the sets of nodes and edges respectively. Each edge $e \in E$ will
be noted by a tuple, $e=(v,w,i)$, indicating that it connects the node $v\in V$
with the node $w \in V$ and that his owner is the agent $i\in N$. Note that
could exist as many edges between a pair of nodes as agents exists. Each edge $e
\in E$ has certain units of capacity associated, $q_e$, and a a fixed activation
cost $c_e$, that the owner of the edge have to pay if the edge is used to route
any commodity. We will note by $OE(v)\subset E$ the subset of edges whose origin
is the node $v\in V$. Similarly, $IE(w)\subset E$ is the subset of edges whose
terminal node is $w\in V$.

Let $\Theta$ be the set of commodities. Each commodity is noted by a tuple,
$(o,t,i)$, where $o\in V$ is its origin node, $t\in V$ its destiny node and $i$
is the agent who owns it. A commodity $(o,t,i)$ ocupies $d_{(o,t,i)}$ units of
capacity of the edges it pass throu, and has an associated revenue
$r_{(o,t,i)}$. The owner of that commodity earns $d_(o,t,i)\cdot r_(o,t,i)$ if
the commodity is succesfully delivered from its origin to its terminal node. The
commodities are unsplitable, i.e., can not be divided between different edges.

We will note by $E^i \subset E$ and $\Theta^i\subset \Theta$ the subsets of
edges and commodities owned by agent $i$.

\subsection{Single agent model}

The objective of an agent $i \in N$ is to maximize her payoff, by routing her
commodities from their origin nodes to their terminal nodes through the edges
she has activated. If the agent does not take part of the collaboration, she
will not have any acces to other agents edges, neither the other agents will
have acces to her edges. Thus, the problem which agent $i\in N$ has to solve can
be modeled as the following ILP, that we call $P_i$, 





\begin{subequations}
    \begin{alignat}{3}
        &  P_i: \hspace{10pt} \max  &  \sum_{(o,t,i)\in \Theta^i} \sum_{e \in IE^i(t)}  f_e^{(o,t,i)} \cdot d_{(o,t,i)} \cdot r_{(o,t,i)} - \sum_{e'\in E^i} u_{e'}\cdot c_{e'} \hspace{20pt} &&   \label{eq:SingleAgentA}
    \end{alignat}
    \begin{alignat}{3}
        & \text{subject to}       & \sum_{e \in IE^i(z)} f_e^{(o,t,i)}-\sum_{e' \in OE^i(z)} f_{e'}^{(o,t,i)} & = 0,                                   && \forall\ z\in V\setminus\{o,t\},\nonumber\\[-1em]
        &                         &                                                                           &                                        && \forall\ (o,t,i)\in\Theta^i,  \label{eq:SingleAgentB}\\[1em]
        &                         & \sum_{e \in OE^i(o)} f_e^{(o,t,i)}                                        & \leq 1,                                && \forall\ (o,t,i)\in \Theta^i, \label{eq:SingleAgentC} \\
        &                         & \sum_{e \in OE^i(t)} f_e^{(o,t,i)}                                        & = 0,                                   && \forall\ (o,t,i)\in \Theta^i, \label{eq:SingleAgentD} \\
        &                         & \sum_{(o,t,i) \in \Theta^i} f_e^{(o,t,i)}\cdot d_{(o,t,i)}                & \leq u_e\cdot q_{(o,t,i)},\hspace{8pt} && \forall\ e \in E^i, \label{eq:SingleAgentE}  \\
        &                         & \sum_{v \in S} \sum_{w \in S} f_{(v,w)}^{(o,t,i)}                         & \leq |S| -1,                           && \forall\ S \subset V, \nonumber\\[-1em]
        &                         &                                                                           &                                        && \forall\ (o,t,i) \in \Theta^i, \label{eq:SingleAgentF}\\[1em]
        &                         & f_e^{(o,t,i)}                                                             & \in \{0,1\},                           && \forall\ e \in E^i,\nonumber\\
        &                         &                                                                           &                                        && \forall\ (o,t,i) \in \Theta^i, \label{eq:SingleAgentG} \\[1em]  
        &                         &  u_e                                                                      & \in \{0,1\},                           && \forall\ e \in E^i 
    \end{alignat}

\end{subequations}

where, $\forall\ (o,t,i)\in \Theta^i,\  \forall\ e \in E$
\[
\begin{array}{rl}
f_e^{(o,t,i)} = & \begin{cases}
    1 & \text{if } (o,t,i) \text{is routed through } e,\\
    0 & \text{otherwise}
\end{cases}  \\[20pt]
u_e = &\begin{cases}
    1 & \text{if } e \text{ is activated},\\
    0 & \text{otherwise}    
\end{cases}
\end{array}
\]

The objective of agent $i$, equation (\ref{eq:SingleAgentA}), is to maximize the
profit generated by the commodities which arrive to their destination nodes
while minimizing the cost associated to the activation of the edges. Constraint
(\ref{eq:SingleAgentB}) ensures that, for any commodity, the flow that enters
and leaves every transit nodes (neither the origin, neither the terminal node of
that commodity) is equal. Constraints (\ref{eq:SingleAgentC}), together with
(\ref{eq:SingleAgentG}), guarantee that each commodity is send at most only once
from its origin node. Constraint (\ref{eq:SingleAgentD}) ensures that none
commodity is send from its terminal node. Constraint (\ref{eq:SingleAgentE})
ensures that the capacity of edges is not exceed. Finally, constraint
(\ref{eq:SingleAgentF}) are subtour elimination constraints.

We will note by $P_i^*$ the optimal solution of $P_i$ and it is the maximal
payoff agent $i$ can obtain by his own, without cooperating with the other agents.

\section{The centralized cooperation models}

In the previous subsection, we have presented ILP that models the problem
of an agent $i\in N$ who works independently. In this section, we present
different centralized cooperative mechanisms that the agents in $N$ could make
use of, in order to find synergies among them and increase their payoffs.

In all our centralized models, we will allocate the benefits of the cooperation
in the following way:
\begin{enumerate}
    \item The revenues generated by every succesfully supplied commodity will be
    allocate to its owner.
    \item The activation cost of every active edge will be payed by its owner.
    \item If a commodity $(o,t,i)$ is routed through an edge $e=(v,w,j)$ and
    $j\not = i$, agent $i$ will make a side payment equal to $d_{(o,t,i)} \cdot
    \frac{c_e}{q_e}$ to the agent $j$. In words, for each commodity sent through
    an edge owned by other agent, the owner of the commodity will pay to the
    owner edge the fraction of the activation cost of that edge equivalent to
    the fraction of the total capacity of the edge that his commodity occupies.
\end{enumerate}

Therefore, for any solution resulting from the cooperation, $\alpha$, we can express the
payoff corresponding to an agent $i\in N$ as


Furthermore, similarly to what is done \cite{VANOVERMEIRE2014125}, we will
integrate the payoff of each agent as contraints of the central planner's
problem, ensuring that the final payoff of each agent is not smaller than what
they could achieve by their own without cooperation. Doing this, we ensure that
the final payoff allocation is \emph{individually rational} \cite{GONZALEZ2010},
i.e., no single agent is interested in leaving the coalition. 
\footnote{We can extend this idea by forcing the solution resulting from the
cooperation to be in the \emph{core}, by adding constraints ensuring that any
the sum of the payoffs of the members of any subcoalition is equal or greater
than what that agents could obtain by leaving the grand coalition and
cooperating only among them. Nevertheless, doing it will increase the complexity
of the models and we consider that it does not add value to the paper.}

In the rest of the section we introduce the three different centralized
cooperative scenarios we propose in this work, that as stated before, differ in the
degree of decision power given to the central planner.


\subsection{Full cooperation scenario}

We start introducing the model where more power is given to the central planner.
In this model, all the commodities and edges of all the agents are given to the
central planner, who has to solve the following ILP,

\begin{subequations}
    \begin{alignat}{3}
        &  \Gamma_F: \max  & \hspace{22pt} \sum_{(o,t,i)\in \Theta} \sum_{e \in IE(t)}  f_e^{(o,t,i)} \cdot d_{(o,t,i)} \cdot r - \sum_{e\in E} u_{e}\cdot c_{e} \hspace{40pt} &&   \label{eq:FullCooperationA}
    \end{alignat}
    \begin{alignat}{3}
        & \text{subject to}       & \sum_{e \in IE(z)} f_e^{(o,t,i)}-\sum_{e' \in OE(z)} f_{e'}^{(o,t,i)} & = 0,                                   && \forall\ z\in V\setminus\{o,t\},\nonumber\\[-1em]
        &                         &                                                                           &                                        && \forall\ (o,t,i)\in\Theta,  \label{eq:FullCooperationB}\\[1em]
        &                         & \sum_{e \in OE(o)} f_e^{(o,t,i)}                                        & \leq 1,                                && \forall\ (o,t,i)\in \Theta, \label{eq:FullCooperationC} \\
        &                         & \sum_{e \in OE(t)} f_e^{(o,t,i)}                                        & = 0,                                   && \forall\ (o,t,i)\in \Theta, \label{eq:FullCooperationD} \\
        &                         & \sum_{(o,t,i) \in \Theta} f_e^{(o,t,i)}\cdot d_{(o,t,i)}                & \leq u_e\cdot q_{(o,t,i)},\hspace{8pt} && \forall\ e \in E, \label{eq:FullCooperationE}  \\
        &                         & \sum_{v \in S} \sum_{w \in S} f_{(v,w)}^{(o,t,i)}                         & \leq |S| -1,                           && \forall\ S \subset V, \nonumber\\[-1em]
        &                         &                                                                           &                                        && \forall\ (o,t,i) \in \Theta, \label{eq:FullCooperationF}\\[1em]
        &                         & f_e^{(o,t,i)}                                                             & \in \{0,1\},                           && \forall\ e \in E,\nonumber\\
        &                         &                                                                           &                                        && \forall\ (o,t,i) \in \Theta, \label{eq:FullCooperationG} \\[1em]  
        &                         &  u_e                                                                      & \in \{0,1\},                           && \forall\ e \in E \\
        &                         & \varphi_i(\Gamma_F)                                                           & \geq P_i^*,     && \forall\ i\in N \label{eq:FullCooperationH}
    \end{alignat}
\end{subequations}

where 

\begin{align}
    \begin{split}
    & \varphi_i(\Gamma_F) = \\
    & = \sum_{(o,t,i)\in \Theta^i} \left[ \sum_{e \in IE^i(o)} f_e^{(o,t,i)} \cdot d_{(o,t,i)} \cdot r_{(o,t,i)} -  \sum_{e\in \Theta^j \colon j\not = i} f_e^{(o,t,i)} \cdot d_{(o,t,i)} \cdot \frac{c_e}{q_e} \right] + \\
    & + \sum_{\substack{(o,t,k) \in \Theta  \colon \\ k \not = i}} \left[\sum_{e \in E^i} f_e^{(o,t,k)} \cdot d_{(o,t,k)} \cdot \frac{c_e}{q_e}\right] - \sum_{e \in E^i} u_e \cdot c_e, \label{eq:FullCooperationPayoff}
    \end{split}
\end{align}

is the payoff allocate to the agent $i\in N$ for any feasible solution
of $\Gamma_F$.

Note that the linear programm that the central planner has to solve in the
\emph{full cooperation} scenario is almost identical to $P_ i$, the ILP that an
agent $i\in N$ has to solve when no cooperating. Both problems only differ in
two aspects. First,  the central planner manages the commodities and edges of
all the agents, $\Theta$ and $E$, while an agent $i$ only has access to his own
assets, $\Theta^i$ and $E^i$. Second, the constraint (\ref{eq:FullCooperationH})
is exclusive to the cooperation scenario. This constraint  guarantees that the
vector of payoffs associated with the optimal solution found by the central
planner, $\Gamma_F^*$, is individually rational. This means, as already
introduced before, that the payoff of every agent $i \in N$, $\varphi_i(\Gamma_F^*)$,
has to be equal or greater than the best payoff that the agent would obtain by his own without
cooperating, $P_i^*$.


\subsection{Partial cooperation scenario}

In the partial cooperation scenario, the central planner has to optimize once
again the flow of all the commodities of all the agents through the network
maximizing the total generated revenue. Nevertheless, the decision of which
edges activate remains as an individual decision of each agent.

In our model, each agent $i \in N$ solves his own $P_i$ problem. Precisely the edges
that are active in the optimal solutions, $P_i^*$, of each agent are the agents
that the central planner can use to route the commodities throught the network.

Let $E_A \subset E$ be the subset of edges such that $\forall e=(v,w,i) \in
E_A$, $e$ is active in $P_i$. Then 

the ILP, $\Gamma_P$, that the central planner has to solve
in the partial cooperation scenario is


\begin{subequations}
    \begin{alignat}{3}
        &  \Gamma_P: \max  & \sum_{(o,t,i) \in \Theta} \left[\sum_{e \in OE_A(o)}  f_e^{(o,t,i)} \cdot d_{(o,t,i)} \cdot r_{(o,t,i)} - \sum_{\substack{e \in E_A^j\colon \\ j\not = i}} f_e^{(o,t,i)} \cdot \frac{c_e}{q_e} \right] &&   \label{eq:Partial2CooperationA} 
    \end{alignat}
    \begin{alignat}{3}
        & \text{subject to}       & \sum_{e \in IE_A(z)} f_e^{(o,t,i)}-\sum_{e' \in OE_A(z)} f_{e'}^{(o,t,i)} & = 0,            && \forall\ z\in V\setminus\{o,t\},\nonumber\\[-1em]
        &                         &                                                                       &                 && \forall\ (o,t,i)\in\Theta,  \label{eq:Partial2CooperationB}\\[1em]
        &                         & \sum_{e \in OE_A(o)} f_e^{(o,t,i)}                                      & \leq 1,         && \forall\ (o,t,i)\in \Theta, \label{eq:Partial2CooperationC} \\
        &                         & \sum_{e \in OE_A(t)} f_e^{(o,t,i)}                                      & = 0,            && \forall\ (o,t,i)\in \Theta, \label{eq:Partial2CooperationD} \\
        &                         & \sum_{(o,t,i) \in \Theta} f_e^{(o,t,i)}\cdot d_{(o,t,i)}              & q_{(o,t,i)}     && \forall\ e \in E_A, \label{eq:Partial2CooperationE_A}  \\
        &                         & \sum_{v \in S} \sum_{w \in S} f_{(v,w)}^{(o,t,i)}                     & \leq |S| -1,    && \forall\ S \subset V, \nonumber\\[-1em]
        &                         &                                                                       &                 && \forall\ (o,t,i) \in \Theta, \label{eq:Partial2CooperationF}\\[1em]
        &                         & f_e^{(o,t,i)}                                                         & \in \{0,1\},    && \forall\ e \in E_A,\nonumber\\
        &                         &                                                                       &                 && \forall\ (o,t,i) \in \Theta, \label{eq:Partial2CooperationG} \\
        &                         & \varphi_i(\Gamma_P)                                                  & \geq P_i^*,     && \forall\ i\in N \label{eq:Partial2CooperationH}
    \end{alignat}
\end{subequations}




\subsection{Residual cooperation}

This is the model among the three in which less decision are left to the central
planner, and it is inspired in the already mentioned framework presented in
\cite{ANUPINDI2001}. 

In the first place, every agent $i\in N$ will indivually solve his corresponding
$P_i$ problem by his own, deciding which edges to activate and how to route his
commodities through that edges. Secondly, each agent will inform the central
planner about which commodities they have not been able to route in an efficient
way, as well as which active edges still have some free capacity. With all this
information, the central planner routes in a efficient way the commodities

\begin{table}[ht!]
    \begin{tabular}{cc|ccc}
        & &      \multicolumn{3}{|c}{Level cooperation} \\\cline{3-5}
        & & Partial-1 & Partial-2 & Full \\ \hline
        AGENT & Activate edges & Yes & Yes & No \\
        '' & Route flow     & Yes & No & No \\\hline
        Central planner & Activate edges & No & No & Yes \\
        '' & Route flow & Yes & Yes & Yes
        \end{tabular}
    \end {table}

\subsubsection*{Partial-1 cooperation}


In the first type of cooperation, what we will call ``Partial-1 cooperation'',
each agent $i\in N$ will solve its own ($P^i$) problem and then pass to the
central planner the demands he was not able to serve and the not used capacity
of the edges he has activated. Will the information from all the agents, the
central planner will try to maximize the profits of the grand coalition, routing
the unserved commodities through the not used capacity of the edges. If a
commodity is route through an edge ow by a different agent, the owner of the
commodity will pay to the owner of the edge the proportional cost of the
fraction of the capacity of th edge he is using.

Therefore, the central planner would be solving the following model:


\begin{subequations}
    \begin{alignat}{3}
        &  \Gamma_1: \max  & \sum_{(o,t,i) \in \Theta} \left[\sum_{e \in OE(o)}  f_e^{(o,t,i)} \cdot d_{(o,t,i)} \cdot r_{(o,t,i)} - \sum_{\substack{e \in E^j\colon \\ j\not = i}} f_e^{(o,t,i)} \cdot \frac{c_e}{q_e} \right] &&   \label{eq:Partial1CooperationA} 
    \end{alignat}
    \begin{alignat}{3}
        & \text{subject to}       & \sum_{e \in IE(z)} f_e^{(o,t,i)}-\sum_{e' \in OE(z)} f_{e'}^{(o,t,i)} & = 0,            && \forall\ z\in V\setminus\{o,t\},\nonumber\\[-1em]
        &                         &                                                                       &                 && \forall\ (o,t,i)\in\Theta,  \label{eq:Partial1CooperationB}\\[1em]
        &                         & \sum_{e \in OE(o)} f_e^{(o,t,i)}                                      & \leq 1,         && \forall\ (o,t,i)\in \Theta, \label{eq:Partial1CooperationC} \\
        &                         & \sum_{e \in OE(t)} f_e^{(o,t,i)}                                      & = 0,            && \forall\ (o,t,i)\in \Theta, \label{eq:Partial1CooperationD} \\
        &                         & \sum_{(o,t,i) \in \Theta} f_e^{(o,t,i)}\cdot d_{(o,t,i)}              & q_{(o,t,i)}     && \forall\ e \in E, \label{eq:Partial1CooperationE}  \\
        &                         & \sum_{v \in S} \sum_{w \in S} f_{(v,w)}^{(o,t,i)}                     & \leq |S| -1,    && \forall\ S \subset V, \nonumber\\[-1em]
        &                         &                                                                       &                 && \forall\ (o,t,i) \in \Theta, \label{eq:Partial1CooperationF}\\[1em]
        &                         & f_e^{(o,t,i)}                                                         & \in \{0,1\},    && \forall\ e \in E,\nonumber\\
        &                         &                                                                       &                 && \forall\ (o,t,i) \in \Theta, \label{eq:Partial1CooperationG} 
    \end{alignat}
\end{subequations}

where $E$ is the set of all the edges, $e$, activated by an agent and with some
not used capacity, $q_e$; and $\Theta$ is the set of all the demands that the
agent left unserved.

We will note by $\Gamma_1^*$ the optimal solution of  $\Gamma_1$, and by
$\varphi_i(\Gamma_1^*)$ the payoff corresponding to agent $i$, with

\begin{align}
    \begin{split}
    \varphi_i(\Gamma_1) = & \sum_{(o,t,i)\in \Theta^i} \left[ \sum_{e \in IE^i(o)} f_e^{(o,t,i)} \cdot d_{(o,t,i)} \cdot r_{(o,t,i)} - \right. \\
                          & \left. - \sum_{e'\in \Theta^j \colon j\not = i} f_e^{(o,t,i)} \cdot d_{(o,t,i)} \cdot \frac{c_{e'}}{q_{e'}} \right] + \\
                          & + \sum_{(o,t,k) \in \Theta \colon k \not = i} \sum_{e'' \in E^i} f_{e''}^{(o,t,k)} \cdot d_{(o,t,k)} \cdot \frac{c_{e''}}{q_{e''}},
    \end{split}
\end{align}

for every $i \in N$. Note that the first sum equals the payoff that agent $i$
obtains from each of her demands, i.e., the revenue she obtains from
subcessfully serving it minus the cost of the commodity being route through
edges of another agents. The second sum equals the ``side payments'' that agent
$i$ obtains from other agents whose commodities are route through her edges.
Thus, the final payoff of each agent $i\in N$ when the Partial-1 cooperation is
used would be $\varphi_i(\Gamma_1^*)+P_i^*$


\subsubsection*{Partial-2 cooperation}

In the second type of cooperation that we study, ``Partial-2 cooperation'', all
the agent will also start by solving their $P^i$ problems. Once they have done
it, they do not route any commodity, but they pass all the commodities to the
central planner, as well as all the edges they activate following their optimal
solution for $(P^i)$. Then, as in ``Partial-1 cooperation'', the central planner
will try to maximize the profits of the Grand Coalition, but also ensuring that
any agent will get a final payoff smaller than what he will optain without
cooperation. Therefore, to the model of ``Partial-1 cooperation'' is added the
following constraint:




Constraint (\ref{eq:Partial2CooperationG}) ensures that the flow is routed in
such a way that every agent ends up with a payoff equal or greater than what
(s)he could optain without cooperation.

\subsubsection*{Full cooperation}


The final type of cooperation we study, ``Full cooperation'', is the case where
all the agent pass to the central planner all their commodities, and let the
central planner to decide which edges to activate and how to route the
commodities. As in ``Partial-2 cooperation'', the central planner must ensure
that any agent end up with a payoff smaller than what (s)he would have obtained
without cooperation.

with 
    \begin{align}
        \begin{split}
        \varphi_i(\Gamma_F) = & \sum_{(o,t,i)\in \Theta^i} \left[ \sum_{e \in IE^i(o)} f_e^{(o,t,i)} \cdot d_{(o,t,i)} \cdot r_{(o,t,i)} - \right. \\
                              & \left. - \sum_{e'\in \Theta^j \colon j\not = i} f_e^{(o,t,i)} \cdot d_{(o,t,i)} \cdot \frac{c_{e'}}{q_{e'}} \right] + \\
                              & + \sum_{(o,t,k) \in \Theta \colon k \not = i} \sum_{e'' \in E^i} f_{e''}^{(o,t,k)} \cdot d_{(o,t,k)} \cdot \frac{c_{e''}}{q_{e''}} -\\
                              & - \sum_{e \in E^i} u_e \cdot c_e,
        \end{split}
    \end{align}
\section{An iterative optimization mechanism}
a
\section{Results}
a
\section{Conclusion}
a

\bibliography{bibliography}

\end{document}