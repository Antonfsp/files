\documentclass[review]{elsarticle}

\usepackage{amsmath,amssymb}
\usepackage{adjustbox}
\usepackage{multirow}
\usepackage{threeparttable}
\usepackage[utf8]{inputenc}
\usepackage{booktabs, caption, makecell}

\usepackage{lineno,hyperref}
\modulolinenumbers[5]
\allowdisplaybreaks


% This avoid a foot note about to which paper the article was submited
\makeatletter
\def\ps@pprintTitle{%
 \let\@oddhead\@empty
 \let\@evenhead\@empty
 \def\@oddfoot{}%
 \let\@evenfoot\@oddfoot}
\makeatother

%%%%%%%%%%%%%%%%%%%%%%%
%% `Elsevier LaTeX' style
\bibliographystyle{elsarticle-num}
%%%%%%%%%%%%%%%%%%%%%%%

\begin{document}

\begin{frontmatter}

\title{Intertemporal logistics collaboration \\[5pt]
                \normalsize{Msc Thesis}}

%% Group authors per affiliation:
\author{Antón de la Fuente suárez-Pumariega}
\address{a.delafuentesuarez-pumariega@student.maastrichtuniverisry.nl }

%% or include affiliations in footnotes:
\author{Maastricht Univeristy}



\begin{abstract}
Here goes the abstract
\end{abstract}

\begin{keyword}
Here\sep go \sep KEY \sep words
\end{keyword}

\end{frontmatter}

% Desactivation of numbering every 15 LINES
%\linenumbers

\section{Introduction}

During the last decades, logistics collaboration, and specially \emph{horizontal
logistics collaboration}, has gained the attention of many researchers, as it
has been proofed to be an effective strategy to improve the logistics chain, for
example, reducing both ecological \cite{BALLOT2010} or economical (find cite)
operational costs. By horizontal logistics collaboration we refer to the
cooperation between several companies or agents, who playing the same role in
the logistics chain, form coalitions or alliances to increase the efficiency of
their logistics operations. For example, in the case of the liner hipping
industry, by allowing other companies to use part of the capacity of the own
ships, increasing the asset utilization ratio \cite{AGARWAL2008175}.


The literature studying the horizontal logistics collaboration between agents
can be divided in two main streams depending on whether the collaboration it is
centralized or not. 

In the case of centralized systems, agents usually share in a pool all or
part of the demands that they have to serve as well as the assets they have
available. These assets can be vehicles in the case of transportation carriers,
but also inventory levels, employees, etc. Then a central planner, taking into
account all the demands and sources shared in the pool by the agents, solves the
problem maximizing the total payoff of the coalition. Finally, the benefits of
the cooperation are shared among the members of the coalition, usually using
some allocation rule which ideally satisfies some desirable conditions in
terms of ``fairness'', as that the resulting allocation is in the \emph{core},
i.e., any subcoalition of the former one could obtain a higher payoff working
independently \cite{GONZALEZ2010}. A well known example of an allocation rule
fulfilling this conditions is the \emph{Shapley value} \cite{SHAPLEY1952}.

Some authors have argued in the centralized systems, the individual objectives
of the agents are not considered in favour of the coalition objective, while
actually the agents remain independent entities \cite{DEFRYN2018891}. Therefore,
these authors propose different mechanisms where both levels of objectives are
considered \cite{DEFRYN20191} \cite{VANOVERMEIRE2014125}.

In the other hand, in the decentralized systems there is not a central planner
who carries out the optimization and then allocates the benefits, but the agents
remain independent decision makers during the whole collaboration process.
Decentralized systems commonly operate based on auctions systems or
inverse-optimization techniques \cite{XIAOZHOU2013}. In the first, agents share
the part of their demands, an the other members of the coalition bid to be the
ones serving that demand on exchange of a recompense from the original owner of
the order. In the other hand, an example of a collaborative system based in a
inverse-optimization mechanism is the proposed in \cite{AGARWAL2008520}. On it,
the authors model a multicommodity flow problem, where agents own certain
fraction of the capacity of the edges of a network. Thus, the agents can
collaborate sharing the capacity of the edges with the members of the coalition,
who have to pay some price for using that edges. How much capacity the agents
own on each edge of the and which are their demand is assumed to be public
information. Using inverse optimization, capacity exchange prices are
found, in such a way that if the agents adopt that prices policy, when each
agent selfishly optimizes his demands through the shared network, the resulting
flow will be the same that if a centralized approach was used. 

Some authors have argued that in the centralized systems, the individual
objectives of the agents are not considered in favour of the coalition
objective, while the agents actually remain independent entities
\cite{DEFRYN2018891}. Therefore, these authors propose different mechanisms where
both levels of objectives are considered through multi-objective approaches
\cite{DEFRYN20191} or by integrating the objectives of the agents as constraints
to the coalition problem \cite{VANOVERMEIRE2014125}.

In \cite{ANUPINDI2001} a novel framework to model decentralized systems is
proposed. Based in the concept of \emph{coopetition} \cite{BRANDENBURGER1996},
the authors proposed a 2-stage mechanism to model the collaboration of agents in
decentralized distribution systems. In the first stage, agents individually
take strategic decisions, in their case, fixing their inventory levels. In the
seconds stage an allocation of the assets is obtained through cooperation, were
agents share their residual demands and inventory with the others, obtaining
extra profits. They propose sufficient conditions under which a Nash equilibrium
in pure strategies exist for the first stage, and for which allocation rule in
the \emph{core} exists for the second stage.



In this paper we will compare the efficiency between a centralized system,
giving to central planner different degrees of decision power, and a
 one, where we substitute the central planner by a platform where
agents can share information, and through an iterative process of optimization
and information sharing, agents can achieve an equilibrium point from which any
of the agents would be interested to deviate from. The problem we
will use as example to implement this models is a multicommodity flow problem,
where each agent have to decide which edges to active on the network at certain
price, to then find an efficient flow through them which maximizes their
payoff.

In section blablbla. In section blalbla. Finally in sectioin blalblalba


\section{Problem statement: A network design - muldicommodity flow problem}

As stated in the previous section, we will use a multicommodity flow problem
with a network design stage to illustrate the different cooperation
mechanisms we study in this paper. In the problem we propose, a simplified version of the one studied in \citep{AGARWAL2008175}, a set of agents have
to first decide which edges they want to activate in a network, with certain
costs associated. Then, each have to route their commodities through the edges
he has activate. The objective of the agents is to maximize the revenues generated by the served commodities and minimize the costs associated with the activation of the edges. This two subproblems are intrinsically linked: the most efficient flow an agent can find for his commodities depends on which edges he has access too, and to know which edges to activate, the agent need to know how he will route his commodities through that edges. 

The agents can collaborate among them sharing the capacity they do not use in their active edges, allowing others to route their commodities through them at some price. 

% On the rest of this section we will introduce the specific characteristics of these problem as well as the notation we will use. We will finish the section modelling the problem that each agent would face in case he does not take part of the collaboration.

Let $N=\{1,...,n\}$ be a set of agents. Let $G=(V,E)$ be a directed graph with
$V$ and $E$ the sets of nodes and edges respectively. 
Let $\Theta$ be the set of commodities. Each commodity is noted by a tuple,
$(o,t,i)$, where $o\in V$ is its origin node, $t\in V$ its destiny node and $i$
is the agent who owns it. A commodity $(o,t,i)$ has a size of $d_{(o,t,i)}$ units and has an associated revenue $r_{(o,t,i)}$. The owner of the commodity $(o,t,i)$ earns $d_(o,t,i)\cdot r_(o,t,i)$ if the commodity is successfully delivered from its origin to its terminal node. The commodities are unsplitable, i.e., can not be divided between different edges. Each edge $e \in E$ will be noted by a tuple, $e=(v,w,i)$, indicating that it connects the node $v\in V$
with the node $w \in V$ and that his owner is the agent $i\in N$. Note that
could exist as many edges between a pair of nodes as agents exists. Each edge $e
\in E$ has certain units of capacity associated $q_e$, and a a fixed activation
cost $c_e$, that the owner of the edge have to pay if he wants to use the edge to route any commodity. We will note by $OE(v)\subset E$ the subset of edges whose origin is the node $v\in V$. Similarly, $IE(v)\subset E$ is the subset of edges whose terminal node is $v\in V$.

We will note by $E^i \subset E$ and $\Theta^i\subset \Theta$ the subsets of
edges and commodities owned by agent $i$.

\subsection{Single agent model}

The objective of an agent $i \in N$ is to maximize his payoff, by routing his
commodities, $\Theta^i$, from their origin nodes to their terminal nodes through the edges $E^i$, he has activated. If the agent does not take part of the collaboration, she will not have any access to other agents edges, neither the other agents can route their commodities through his edges. Thus, the problem which agent $i\in N$ has to solve can be modelled as the following ILP, that we call $P_i$, 

\begin{subequations}
    \begin{alignat}{3}
        &  P_i: \hspace{10pt} \max  &  \sum_{(o,t,i)\in \Theta^i} \sum_{e \in IE^i(t)}  f_e^{(o,t,i)} \cdot d_{(o,t,i)} \cdot r_{(o,t,i)} - \sum_{e\in E^i} u_{e'}\cdot c_{e} \hspace{20pt} &&   \label{eq:SingleAgentA}
    \end{alignat}
    \begin{alignat}{3}
        & \text{subject to}       & \sum_{e \in IE^i(z)} f_e^{(o,t,i)}-\sum_{e \in OE^i(z)} f_{e}^{(o,t,i)} & = 0,                                   && \forall\ z\in V\setminus\{o,t\},\nonumber\\[-1em]
        &                         &                                                                           &                                        && \forall\ (o,t,i)\in\Theta^i,  \label{eq:SingleAgentB}\\[1em]
        &                         & \sum_{e \in OE^i(o)} f_e^{(o,t,i)}                                        & \leq 1,                                && \forall\ (o,t,i)\in \Theta^i, \label{eq:SingleAgentC} \\
        &                         & \sum_{e \in OE^i(t)} f_e^{(o,t,i)}                                        & = 0,                                   && \forall\ (o,t,i)\in \Theta^i, \label{eq:SingleAgentD} \\
        &                         & \sum_{(o,t,i) \in \Theta^i} f_e^{(o,t,i)}\cdot d_{(o,t,i)}                & \leq u_e\cdot q_{(o,t,i)},\hspace{8pt} && \forall\ e \in E^i, \label{eq:SingleAgentE}  \\
        &                         & \sum_{v \in S} \sum_{w \in S} f_{(v,w)}^{(o,t,i)}                         & \leq |S| -1,                           && \forall\ S \subset V, \nonumber\\[-1em]
        &                         &                                                                           &                                        && \forall\ (o,t,i) \in \Theta^i, \label{eq:SingleAgentF}\\[1em]
        &                         & f_e^{(o,t,i)}                                                             & \in \{0,1\},                           && \forall\ e \in E^i,\nonumber\\
        &                         &                                                                           &                                        && \forall\ (o,t,i) \in \Theta^i, \label{eq:SingleAgentG} \\[1em]  
        &                         &  u_e                                                                      & \in \{0,1\},                           && \forall\ e \in E^i 
    \end{alignat}

\end{subequations}

where, $\forall\ (o,t,i)\in \Theta^i$ and  $\forall\ e \in E$,
\[
\begin{array}{rl}
f_e^{(o,t,i)} = & \begin{cases}
    1 & \text{if } (o,t,i) \text{ is routed through } e,\\
    0 & \text{otherwise}
\end{cases}  \\[20pt]
u_e = &\begin{cases}
    1 & \text{if } e \text{ is activated},\\
    0 & \text{otherwise}    
\end{cases}
\end{array}
\]

The objective of agent $i$, equation (\ref{eq:SingleAgentA}), is to maximize the
profit generated by the commodities which arrive to their destination nodes
while minimizing the cost associated to the activation of the edges. Constraints
(\ref{eq:SingleAgentB}) ensure that, for any commodity, the flow that enters
and leaves every transit nodes (neither the origin, neither the terminal node of
that commodity) is equal. Constraints (\ref{eq:SingleAgentC}), together with
(\ref{eq:SingleAgentG}), guarantee that each commodity is send at most only once
from its origin node. Constraints (\ref{eq:SingleAgentD}) ensure that no
commodity is send from its terminal node to any other node. Constraint (\ref{eq:SingleAgentE})
ensures that the capacity of edges is not exceed. Finally, constraints
(\ref{eq:SingleAgentF}) are subtour elimination constraints. We have included these constraints in our model because, first of all, it makes sense to assume that an agent would not want to create subtours when routing his commodities, as they would mean that some commodities would occupy unnecessary units of capacity in some edges. Note that his can be possible because once an edge is activated, there is no extra costs for routing more commodities through it. Furthermore, without the subtour elimination constraints, the other constraints are not enough to ensure that all the feasible solution of the ILP are admissible, since the flow of a commodity could contain a subtour disconneted from the origin and terminal nodes of that commodity, as we illustrate in Figure [Include figure!!!!!]. [Or maybe include cost for routing commodities through edges?]

We will note by $P_i^*$ the optimal solution of $P_i$ and it is the maximal
payoff agent $i$ can obtain by his own, without cooperating with the other agents.

\section{The centralized cooperation models}

In the previous subsection, we have presented an ILP that models the network design - multicommodity flow problem of an agent $i\in N$ who works alone. In this section, we present different centralized cooperative scenarios, where the agents in $N$ collaborative in different ways in order to find synergies among them and increase their payoffs.

In all the three different scenarios we propose, the way in which agents is sharing the capacity of the edges they have activate with other agents, what could allow to some agents to serve commodities in a cheaper way to some agents, and to compensate the cost of the activation of some edges to others. The three scenarios also have in common the way in which the agents share the benefits generated by the collaboration, what is done as follows
\begin{enumerate}
    \item The revenues generated by any served commodity are
    allocated to its owner.
    \item The activation cost of any active edge is paid by its owner.
    \item If a commodity $(o,t,i)\in \Theta$ is routed through an edge $e=(v,w,j)		\in E$ and $j\not = i$, agent $i$ makes a side payment equal to $d_{(o,t,i)})		\cdot\frac{c_e}{q_e}$ to the agent $j$. In words, for each commodity sent 		through an edge owned by other agent, the owner of the commodity will pay to 		the owner of the edge the fraction of the activation cost of that edge 			equivalent 	to the fraction of the total capacity of the edge that his 			commodity occupies.
\end{enumerate}

In the rest of the section we introduce the three different centralized
cooperative scenarios we propose in this work.


\subsection{Full cooperation scenario}

We start introducing the model where more power is given to the central planner.
In this model, all the commodities and edges of all the agents are given to the
central planner, who has to solve the following ILP,

\begin{subequations}
    \begin{alignat}{3}
        &  \Gamma_F: \max  & \hspace{22pt} \sum_{(o,t,i)\in \Theta} \sum_{e \in IE(t)}  f_e^{(o,t,i)} \cdot d_{(o,t,i)} \cdot r - \sum_{e\in E} u_{e}\cdot c_{e} \hspace{40pt} &&   \label{eq:FullCooperationA}
    \end{alignat}
    \begin{alignat}{3}
        & \text{subject to}       & \sum_{e \in IE(z)} f_e^{(o,t,i)}-\sum_{e' \in OE(z)} f_{e'}^{(o,t,i)} & = 0,                                   && \forall\ z\in V\setminus\{o,t\},\nonumber\\[-1em]
        &                         &                                                                           &                                        && \forall\ (o,t,i)\in\Theta,  \label{eq:FullCooperationB}\\[1em]
        &                         & \sum_{e \in OE(o)} f_e^{(o,t,i)}                                        & \leq 1,                                && \forall\ (o,t,i)\in \Theta, \label{eq:FullCooperationC} \\
        &                         & \sum_{e \in OE(t)} f_e^{(o,t,i)}                                        & = 0,                                   && \forall\ (o,t,i)\in \Theta, \label{eq:FullCooperationD} \\
        &                         & \sum_{(o,t,i) \in \Theta} f_e^{(o,t,i)}\cdot d_{(o,t,i)}                & \leq u_e\cdot q_{(o,t,i)},\hspace{8pt} && \forall\ e \in E, \label{eq:FullCooperationE}  \\
        &                         & \sum_{v \in S} \sum_{w \in S} f_{(v,w)}^{(o,t,i)}                         & \leq |S| -1,                           && \forall\ S \subset V, \nonumber\\[-1em]
        &                         &                                                                           &                                        && \forall\ (o,t,i) \in \Theta, \label{eq:FullCooperationF}\\[1em]
        &                         & f_e^{(o,t,i)}                                                             & \in \{0,1\},                           && \forall\ e \in E,\nonumber\\
        &                         &                                                                           &                                        && \forall\ (o,t,i) \in \Theta, \label{eq:FullCooperationG} \\[1em]  
        &                         &  u_e                                                                      & \in \{0,1\},                           && \forall\ e \in E, \\
        &                         & \varphi_i(\Gamma_F)                                                           & \geq P_i^*,     && \forall\ i\in N; \label{eq:FullCooperationH}
    \end{alignat}
\end{subequations}

where 
\begin{equation}
    \begin{split}
    & \varphi_i(\Gamma_F) =\label{eq:FullCooperationPayoff} \\
    & = \sum_{(o,t,i)\in \Theta^i} \left[ \sum_{e \in IE^i(o)} f_e^{(o,t,i)} \cdot d_{(o,t,i)} \cdot r_{(o,t,i)} -  \sum_{e\in \Theta^j \colon j\not = i} f_e^{(o,t,i)} \cdot d_{(o,t,i)} \cdot \frac{c_e}{q_e} \right] + \\
    & + \sum_{\substack{(o,t,k) \in \Theta  \colon \\ k \not = i}} \left[\sum_{e \in E^i} f_e^{(o,t,k)} \cdot d_{(o,t,k)} \cdot \frac{c_e}{q_e}\right] - \sum_{e \in E^i} u_e \cdot c_e, 
    \end{split}
\end{equation}

is the payoff allocate to the agent $i\in N$ for any feasible solution
of $\Gamma_F$.

Note that the model that the central planner has to solve in the
\emph{full cooperation} scenario is almost identical to $P_ i$, the ILP that an
agent $i\in N$ has to solve when no cooperating. Both problems only differ in
two aspects:
\begin{enumerate}[(a)]
	\item The central planner manages the commodities and edges of
all the agents, $\Theta$ and $E$, while without cooperation an agent $i$ only has access to his own commodities and edges, $\Theta^i$ and $E^i$.
	\item Second, the constraint (\ref{eq:FullCooperationH}) is exclusive to the cooperation scenario. Similarly to what is done in \cite{VANOVERMEIRE2014125}, we integrate the final payoffs of the agents in the constraints of the central planner's problem. Thus, these constraints ensure that the final payoff of any agent $i\in N$, $\varphi_i$, is greater or equal than what that agent could achieve without cooperation, i.e., $P_i^*$.  Doing so we can say that the final payoff allocation is \emph{individually rational} \cite{GONZALEZ2010}, i.e., no single agent is interested in leaving the coalition unilaterally.  \footnote{We can extend this idea by forcing the solution resulting from the cooperation to be in the \emph{core}, by adding constraints ensuring that any the sum of the payoffs of the members of any subcoalition is equal or greater than what that agents could obtain by leaving the grand coalition and cooperating only among them. Nevertheless, doing it will increase the complexity of the models and we consider that it does not add value to the paper.}
\end{enumerate}


In equation (\ref{eq:FullCooperationPayoff}) we define how the final payoff of an agent $i\in N$, is computed: the first term is the sum of the profit created by the commodities of the agent, which for each commodity is equal to the revenue it
generates when it is served minus the cost the agent has to pay to other agents
if the commodity passes through edges which does not belong to him. The second
sum is the payments that other agents pays to him when routing their commodities
through his edges. The last term is the sum of the costs of the active edges.


\subsection{Partial cooperation scenario}

In the partial cooperation scenario, the central planner has to optimize once
again the flow of all the commodities of all the agents through the network,
maximizing the total generated revenue. Nevertheless, the decision of which
edges to activate remains as an individual decision of each agent.


We propose that, in a first stage, each agent $i \in N$ solves his own $P_i$ problem. Then, in a second stage, the same edges that are active in the optimal solutions of each agent individual ILP, $P_i^*$, are the agents that the central planner can use to route the commodities through the network.

Let $E_A \subset E$ be the subset of edges such that $\forall e=(v,w,i) \in
E_A$, $e$ is active in $P_i^*$. Then the ILP, $\Gamma_P$, that the central
planner has to solve in the partial cooperation scenario is


\begin{subequations}
    \begin{alignat}{3}
        &  \Gamma_P: \max  & \sum_{(o,t,i) \in \Theta} \left[\sum_{e \in OE_A(o)}  f_e^{(o,t,i)} \cdot d_{(o,t,i)} \cdot r_{(o,t,i)} - \sum_{\substack{e \in E_A^j\colon \\ j\not = i}} f_e^{(o,t,i)} \cdot \frac{c_e}{q_e} \right] &&   \label{eq:Partial2CooperationA} 
    \end{alignat}
    \begin{alignat}{3}
        & \text{subject to}       & \sum_{e \in IE_A(z)} f_e^{(o,t,i)}-\sum_{e' \in OE_A(z)} f_{e'}^{(o,t,i)} & = 0,            && \forall\ z\in V\setminus\{o,t\},\nonumber\\[-1em]
        &                         &                                                                       &                 && \forall\ (o,t,i)\in\Theta,  \label{eq:Partial2CooperationB}\\[1em]
        &                         & \sum_{e \in OE_A(o)} f_e^{(o,t,i)}                                      & \leq 1,         && \forall\ (o,t,i)\in \Theta, \label{eq:Partial2CooperationC} \\
        &                         & \sum_{e \in OE_A(t)} f_e^{(o,t,i)}                                      & = 0,            && \forall\ (o,t,i)\in \Theta, \label{eq:Partial2CooperationD} \\
        &                         & \sum_{(o,t,i) \in \Theta} f_e^{(o,t,i)}\cdot d_{(o,t,i)}              &\leq q_e     && \forall\ e \in E_A, \label{eq:Partial2CooperationE_A}  \\
        &                         & \sum_{v \in S} \sum_{w \in S} f_{(v,w)}^{(o,t,i)}                     & \leq |S| -1,    && \forall\ S \subset V, \nonumber\\[-1em]
        &                         &                                                                       &                 && \forall\ (o,t,i) \in \Theta, \label{eq:Partial2CooperationF}\\[1em]
        &                         & f_e^{(o,t,i)}                                                         & \in \{0,1\},    && \forall\ e \in E_A,\nonumber\\
        &                         &                                                                       &                 && \forall\ (o,t,i) \in \Theta, \label{eq:Partial2CooperationG} \\
        &                         & \varphi_i(\Gamma_P)                                                  & \geq P_i^*,     && \forall\ i\in N \label{eq:Partial2CooperationH}
    \end{alignat}
\end{subequations}

Consequently to the above introduced, note that this model differs from the previously present in this work in the absence of a decision variable for activation of the edges, since this decision is previously make by the agents and not left to the central planner. Therefore equation (\ref{eq:FullCooperationPayoff}) no longer holds in this cooperation scenario. Nevertheless, the final payoff of each agent in the partial cooperation scenario can be easily rewritten as follows

\begin{equation}
    \begin{split}
    & \varphi_i(\Gamma_R) =\label{eq:PartialCooperationPayoff} \\
    & = \sum_{(o,t,i)\in \Theta^i} \left[ \sum_{e \in IE_A^i(o)} f_e^{(o,t,i)} \cdot d_{(o,t,i)} \cdot r_{(o,t,i)} -  \sum_{\substack{e\in E^j \colon\\ j\not = i}} f_e^{(o,t,i)} \cdot d_{(o,t,i)} \cdot \frac{c_e}{q_e} \right] + \\
    & + \sum_{\substack{(o,t,k) \in \Theta  \colon \\ k \not = i}} \left[\sum_{e \in E_A^i} f_e^{(o,t,k)} \cdot d_{(o,t,k)} \cdot \frac{c_e}{q_e}\right] - \sum_{e \in E_A^i} c_e.
    \end{split}
\end{equation}

\subsection{Residual cooperation}

Among the three centralized systems we propose in this paper, the
\emph{residual cooperation scenario}, which is inspired in the already mentioned framework presented in \cite{ANUPINDI2001}, is the one where less decision power if left to the central planner.

As in the partial cooperation scenario, every agent $i\in N$ will individually solve his corresponding $P_i$ problem, deciding which edges to activate and how to route his commodities through that edges. But this time, the agents actually implement this solution, and only inform the central planner about which commodities they have not been able to route in an efficient way, as well as which active edges still have some free capacity. With all this information, the central planner routes in a efficient way the residual commodities of the agents, $\Theta_R$, through the residual capacities of the active edges,$E_R$. For each edge $e \in E_R$ we will note by $q_e^R$ the residual capacity of that edge, i.e., the units of capacity that the owner of the edge has not used. Thus, the central planner has to solve the following ILP, that we call $\Gamma_R$,


\begin{subequations}
    \begin{alignat}{3}
        &  \Gamma_R: \max  & \sum_{(o,t,i) \in \Theta_R} \left[\sum_{e \in OE_R(o)}  f_e^{(o,t,i)} \cdot d_{(o,t,i)} \cdot r_{(o,t,i)} - \sum_{\substack{e \in E_R^j\colon \\ j\not = i}} f_e^{(o,t,i)} \cdot \frac{c_e}{q_e} \right] &&   \label{eq:Partial1CooperationA} 
    \end{alignat}
    \begin{alignat}{3}
        & \text{subject to}       & \sum_{e \in IE_R(z)} f_e^{(o,t,i)}-\sum_{e' \in OE_R(z)} f_{e'}^{(o,t,i)} & = 0,            && \forall\ z\in V\setminus\{o,t\},\nonumber\\[-1em]
        &                         &                                                                       &                 && \forall\ (o,t,i)\in\Theta_R,  \label{eq:Partial1CooperationB}\\[1em]
        &                         & \sum_{e \in OE_R(o)} f_e^{(o,t,i)}                                      & \leq 1,         && \forall\ (o,t,i)\in \Theta_R, \label{eq:Partial1CooperationC} \\
        &                         & \sum_{e \in OE_R(t)} f_e^{(o,t,i)}                                      & = 0,            && \forall\ (o,t,i)\in \Theta_R, \label{eq:Partial1CooperationD} \\
        &                         & \sum_{(o,t,i) \in \Theta_R}
        f_e^{(o,t,i)}\cdot d_{(o,t,i)}              & \leq q_e^R     &&
        \forall\ e \in E_R, \label{eq:Partial1CooperationE_R}  \\
        &                         & \sum_{v \in S} \sum_{w \in S} f_{(v,w)}^{(o,t,i)}                     & \leq |S| -1,    && \forall\ S \subset V, \nonumber\\[-1em]
        &                         &                                                                       &                 && \forall\ (o,t,i) \in \Theta_R, \label{eq:Partial1CooperationF}\\[1em]
        &                         & f_e^{(o,t,i)}                                                         & \in \{0,1\},    && \forall\ e \in E_R,\nonumber\\
        &                         &                                                                       &                 && \forall\ (o,t,i) \in \Theta_R, \label{eq:Partial1CooperationG} 
    \end{alignat}
\end{subequations}

In this scenario, it is guaranteed by design of the collaboration
mechanism that each agent end up the cooperation with a payoff greater or equal than the payoff they could obtain by their own, as they always first implement the best solution they can achieve without cooperating, and therefore the cooperating mechanism can only increase their payoffs with respect to that value.

In table \ref{tb:summarycentralizedmodels} we summarize the differences between the three different centralized cooperative scenarios we have proposed in this section, indicating which decision are left to the agents and which decision are made by the central planner.

\begin{table}[ht!]
	\centering
	\caption{Here goes the caption \label{tb:summarycentralizedmodels}}
    \begin{threeparttable}
        \begin{tabular}{ccccc}
            & &      \multicolumn{3}{c}{Coop. centralized scenarios} \\\cline{3-5}
            & & Full & Partial & Residual \\ \hline
            \multirow{2}{*}{Agent} & Activate edges & No & Yes & Yes \\
            & Route flow     & No & No & Yes \\\hline
            \multirow{2}{*}{Central planner} & Activate edges & Yes & No & No \\
            & Route flow & Yes & Yes & Yes\tnote{*} \\\hline\hline
        \end{tabular}
    \begin{tablenotes}\footnotesize
        \item[*] Only the residual commodities through the residual capacities
        of the active edges.
        \end{tablenotes}
    \end{threeparttable}
    \end {table}

\section{A decentralized cooperation mechanism}

In this section we propose a decentralized cooperation mechanism, where all the decisions are left to the agents, which will make use of a information platform to interact with the other agents.

\section{Results}

\section{Conclusion}


\bibliography{bibliography}

\end{document}